
%----------------------------------------------------------------------------------------
%	REQUIRED PACKAGES
%----------------------------------------------------------------------------------------

\usepackage[
nochapters, % Turn off chapters since this is an article        
beramono, % Use the Bera Mono font for monospaced text (\texttt)
eulermath,% Use the Euler font for mathematics
pdfspacing, % Makes use of pdftex’ letter spacing capabilities via the microtype package
dottedtoc % Dotted lines leading to the page numbers in the table of contents
]{classicthesis} % The layout is based on the Classic Thesis style

\usepackage{arsclassica} % Modifies the Classic Thesis package

\usepackage[T1]{fontenc} % Use 8-bit encoding that has 256 glyphs

\usepackage[utf8]{inputenc} % Required for including letters with accents

\usepackage{graphicx} % Required for including images
\graphicspath{{Figures/}} % Set the default folder for images

\usepackage{enumitem} % Required for manipulating the whitespace between and within lists

\usepackage{lipsum} % Used for inserting dummy 'Lorem ipsum' text into the template

\usepackage{subfig} % Required for creating figures with multiple parts (subfigures)

\usepackage{amsmath,amssymb,amsthm} % For including math equations, theorems, symbols, etc

\usepackage{varioref} % More descriptive referencing

\usepackage[german]{babel}

\usepackage{hyperref}

\usepackage{csquotes}

\usepackage{tabularx}

\usepackage{setspace}

% listings
\usepackage{listings}

\lstdefinestyle{mStyle}{
    basicstyle=\linespread{1.0}\normalsize\ttfamily,
    keepspaces=false,
    keywordstyle=\color{Fuchsia},
    identifierstyle=\color{MidnightBlue},
    commentstyle=\color{Green},
    stringstyle=\color{Mahogany},
    columns=flexible
}

%---------------------------------------------------------
%   HYPERREF SETTINGS
%---------------------------------------------------------
\def\figureautorefname{Abbildung}
\def\sectionautorefname{Abschnitt}
\def\subsectionautorefname{Unterabschnitt}
\def\subsubsectionautorefname{Unterunterabschnitt}


%----------------------------------------------------------------------------------------
%	THEOREM STYLES
%---------------------------------------------------------------------------------------

\theoremstyle{definition} % Define theorem styles here based on the definition style (used for definitions and examples)
\newtheorem{definition}{Definition}

\theoremstyle{plain} % Define theorem styles here based on the plain style (used for theorems, lemmas, propositions)
\newtheorem{theorem}{Theorem}

\theoremstyle{remark} % Define theorem styles here based on the remark style (used for remarks and notes)
